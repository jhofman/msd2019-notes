%----------------------------------------
% Write your notes here
%----------------------------------------
\section{Part 1}
\subsection{Conduct 5\% margin of error}

\textbf{5\% margin} margin of error means 5\% standard error or 5\% standard deviation in sampling distribution for the mean.
\newline\textbf{Given 5\%} margin of error, 100 samples are needed. However,if the standard error is large such as 10\%, less samples will need. Following R codes has provided to demonstrate.
\lstset{language=R,
    basicstyle=\small\ttfamily,
    stringstyle=\color{DarkGreen},
    otherkeywords={0,1,2,3,4,5,6,7,8,9},
    morekeywords={TRUE,FALSE},
    deletekeywords={data,frame,length,as,character},
    keywordstyle=\color{blue},
    commentstyle=\color{DarkGreen},
}
\begin{lstlisting}
# flip a coin n times 
p=0.5
rbinom(n,1,p)
#estimate p by measuring the fraction of heads
mean(rbinom(n,1,p)
#repeat this 100,000 times
phat<-replicate(1e5,mean(rbinom(n,1,p))
#look at a histogram of the estimates
hist(phat)
#compute the standard deviation of the estimates
sd(phat)
\end{lstlisting}

\includegraphics[width=12cm]{figures/ddg}


\subsection{Apply Central Limit Theorem to Binomial Distribution}



\begin{equation} Var(\dfrac{1}{n} \sum_{i=0}^n X{i}) =\dfrac{1}{n^2}Var(\sum_{i=0}^n X{i}) = \dfrac{1}{n^2}(\sum_{i=0}^n Var (X{i}))=\dfrac{p(1-p)}{n}
\end{equation}
\newline As a result, the number of sampling can be calculated by following way
\begin{equation}
S=SD(\dfrac{1}{n} \sum_{i=0}^n X{i})=\sqrt{\dfrac{p(1-p)}{n}}
\end{equation}

\begin{equation}
n=\dfrac{p(1-p)}{S^2}
\end{equation}

\section{Part 2}
\subsection{Why Counting}

\textbf{Traditionally} difficult to obtain reliable estimates due to small sample sizes or sparsity.
\newline for example, (100age x 2sex x 5race x 3 party)=3,000 groups
As we know, 100 samples are needed to conduct 5\% margin error. For 3,000 groups, we need 100x3,000=300,000 samples. Apparently, it is almost impossible to conduct such experiment.
\newline\textbf{Potential solution}
\newline1.Combine large observations into fewer groups,which can cause of missing other important info. 
\newline2.Come up with more sophisticated methods generated by small samples.
\newline3.\textbf{Obtain larger data samples } by other ways and then count and divide to make estimates through its relative frequencies.
\newline\textbf{Good and bad of large data }
\newline\textbf{Pros:} move away from complicated and complex models generated by small samples to simpler models on large samples. 
\newline\textbf{Cons:} Computationally challenging at large data.
\subsection{Learning to count}
\textbf{Claim:} Solving the counting problem at scale enables you to investigate many interesting questions in the social sciences.
\newline\textbf{R functions}:
\newline Split: Arrange observations into groups of interest.
\newline Apply: Compute distributions and statistics within each group
\newline Combine: Collect results across groups.
\newline\textbf{Examples:}

\begin{tabular}{cc}
Group & Value \\
a & 2 \\
b & 3 \\
a & 4 \\
c & 10 \\
c & 12 \\
b & 9
\end{tabular}

\noindent\textbf{Several} ways to approach the solution of finding the average per group.
\newline1. First, find all a's and make a list of values such as (2+4)/2=3 for a's. Then, repeat for each group. With G*N steps and N space.
\newline2.Run through observations and create list for each group.Then compute average each list. With N steps and N space.
\newline3.Create list for each group and keep running average instead of list.
\newline\textbf{O(n*\ logn)}: time for binary search and sorting algorithms.\\
\begingroup \centering \textbf{Movielens plots}
\includegraphics{figures/wee}\endgroup 
\newline According to the graph, majority rankings are from the most popular movies.

\begingroup\centering
\includegraphics{figures/ddd}\\
\endgroup
\noindent Compute median movie rank as user eccentricity. However, when encountering a large data set, median is nearly impossible to find.
\newline\textbf{Examples:}
\newline Ex1.
\newline Take all your movies and look at popularity rank such as following
(3,7,100,120,9800)
\newline The median rank as well as eccentricity is 100.
\newline Ex2
\newline
\begin{tabular}{cccc}
user & movie id & rating & ranking \\
1 & 37 & 1.5 & 8,000 \\
1 & 43 & 4.5 & 10 \\
1 & 2 & 3.0 & . \\
. & . & . & . \\
. & . & . & . \\
. & . & . & . \\
2 & 37 & . & 8,000
\end{tabular}

\section{Part 3}
\subsection{The group-by operation}
\textbf{Usually} large data sets need large memory store. 
\begingroup\centering
\includegraphics{figures/dda}\\
\endgroup
\noindent\textbf{Combinable}: means that all data is needed to compute median.
\newline\textbf{Streaming}: is required when full data-set exceeds available memory. It reads data one observation at a time, storing only needed state. It is also useful for computing a subset of within-group statistics with a limited memory footprint such as min, mean, variance but not median, requiring complete data set to compute.
\newline\textbf{Median rating} are used by both Netflix and YouTube.\\
\textbf{Mean rating} is also utilized by YouTube, using streaming to compute combinable statistics.
\newline\textbf{uniq-c} in R: c(a,b,a,a,b,c) to c(a,b,a,b,c). Only delete next repeated occurrence.
\section{Part 4}
\subsection{Practice of Shell}
\textbf{Shell} in Mac OS is called Terminal. \\
If you use Windows, you can try the built in bash/Ubuntu shell on Windows 10 or you can install it which includes bash and a terminal application by default. Linux also includes a working shell and terminal.\\
\textbf{Useful} commands in terminal: curl -o [short name] [URL].
